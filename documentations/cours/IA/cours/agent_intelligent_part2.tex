\documentclass{article}
\usepackage[utf8]{inputenc} % un package
\usepackage[T1]{fontenc}      % un second package
\usepackage[francais]{babel}  % un troisième package


\title{ Cours Agent Intelligent Partie 2}
\author{Jimmy Vogel}
\date{19 mars 2013}

\begin{document}
\maketitle
\newpage
\tableofcontents
\newpage

\section{Rappel PEAS}
\subsection{Exemple}

\paragraph{P(mesure de performance)} 
\begin{itemize}
	\item trouver l'or +1000
	\item mort -1000
	\item -1 par déplacement
	\item -10 pour chaque flèche tiré
\end{itemize}
\paragraph{E(environnement)}
\begin{itemize}
	\item odeur désagréable dans les cases adjacentes au Wumpus
	\item courant d'air dans les cases adjacentes au puits
	\item éclat dans la case contenant l'or
\end{itemize}
\paragraph{A(Actions)}
\begin{itemize}
	\item se déplacer a gauche, droite, en haut, en bas
	\item tirer devant soit
\end{itemize}
\paragraph{S(Sensors)}
\begin{itemize}
	\item sentir l'air
	\item sentir l'odeur du Wumpus
\end{itemize}

\subsection{Cas différents}
\paragraph{}
Dans l'environnement il faut préciser si on est dans un environnement mono-agent.
Ici on aurait pu préciser que le monstre ne peut se déplacer.
Il faut aussi préciser que la taille de la matrice est limité.

\newpage

\section{Systèmes experts}

\subsection{Definition}

\paragraph{}

\begin{itemize}
	\item Un système expert utilise des connaissances spécifiques à un domaine pour fournir des conseils ou des solutions à des problèmes.
  	\item Des connaissances d'un domaine sont présentées dans une base de connaissances (par exemple base de règles).
	\item Simuler le raisonnement de l'expert humain: moteur d'inférence
	\item La performance d'un système expert dépend essentiellement de ses connaissances, moins du moteur d'infférence
	\item Un système expert doit être riche en connaissances : Knowledge is power
	\item Comme un expert humain, un système expert
	\begin{itemize}
		\item se spécialise dans un domaine
		\item enrichit ses connaissances avec des expériences
	\end{itemize}
\end{itemize}

\subsection{Domaines}

\paragraph{}
\begin{itemize}
	\item Interprétation: former des conclusion de haut niveau à partir de données brutes
	\item Prédiction: trouver des consséquences probables des situations données
	\item Diagnostique: déterminer la cause du mal fonctionnement dans des situations complexes à partir des symptômes observables.
	\item Configuration : construire une configuratino des composants pour atteindre des objectifs de performance tout en satisfaisant des contraintes de configuration.
	\item Planification: déterminer une suite d'actions pour arriver à un ensemble d'objectifs.
	\item Surveillance: comparer des comportements d'un système par rapport au comportement souhaité.
	\item Contrôle: contrôler le comportement d'un environnement complexe.
	\item etc.
\end{itemize}

\newpage

\subsection{Architecture}

\paragraph{Base de connaissances}
\begin{itemize}
	\item Les connaissances générales et spécifiques du domaine.
	\item connaissance sous forme de règle Si alors pour des systèmes à base de règles.
	\item des éditeurs à base de connaissances permet de réprésenter les connaissances sous forme facile à accéder.
\end{itemize}

\paragraph{Moteur inférence}
\begin{itemize}
	\item effectuer le raisonnement pour tirer les conséquences impliquées par la connaissance incluse dans le système.
\end{itemize}

\paragraph{Interface utilisateur}
Le mieux est d'avoir la possibilité pour l'utilisateur d'utiliser une langue naturel.
A partir du traitement du langage naturel, obtenir des connaissances ou vérifier des connaissances.
Donne l'impression que l'utilisateur parle avec un humain.

\paragraph{Exigences}
\begin{itemize}
	\item Raisonnement correct
	\item Raisonnement ouvert à l'inspection
	\item Capacité d'explication des choix et des décisions pris
\end{itemize}

\paragraph{Construction}
Aussi des modules pour la création et la gestion d'un système expert peuvent être fournis ou réutilisés:
\begin{itemize}
	\item Jess en Java. 
	\item CLIPS de la Nasa en C.
\end{itemize}
La construction de la base de connaissance est la partie difficile. Elle dépend des domaines.

\newpage

\section{Logique propositionnelle}
\subsection{Definition}
\paragraph{}
La logique des propositions permet d'exprimer
\begin{itemize}
	\item des faits sur le monde
	\item des négations
	\item des conjonctions et des disjonctions
	\item des phrases avec conséquence logique
\end{itemize}
\paragraph{}
Une proposition est une expression(phrase) à propos du monde qui est soit vraie soit fausse. 

\paragraph{}
Elements de base:
\begin{itemize}
	\item sysboles de propositions
	\item phrases spéciales: vrai , faux
	\item opérateurs
\end{itemize}

\subsection{Exemple}
Nous allons réutiliser le problème du Wumpus:
\begin{itemize}
	\item Si la case (2,3) a un puits alors il  y a un courant d'air dans les cases adjacentes.
		$P_{23} \rightarrow C_{13} \land C_{33} \land C_{22} \land C_{24}$
	\item S'il y a du courant d'air dans la case (1,2) alors il doit y avoir un puits dans une case adjacente.
		$C_{12} \rightarrow P_{11} \lor  P_{22} \lor P_{13}$
	\item 1er ordre: $P(i,j) \land adj(i_{1}, j_{1}, i, j) \rightarrow C(i_{1}, j_{1})$
\end{itemize}

\subsection{Rappel}
Voir les lois de négation, d'implication, de DeMorgan ...etc \\
Voir l'algorithme de résolution.

\subsection{Soit}
\paragraph{}
\begin{itemize}
	\item Résolution: connaissances sous formes de clauses
	\item Chainages avant et arrière: connaissances sous forme de clauses de Horn (règles).
\end{itemize}

\subsection{}
\paragraph{}


\end{document}